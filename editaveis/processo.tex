\chapter[Metodologia e Processo]{Metodologia e processo}
\section{Metodologia de desenvolvimento}
Com o passar do tempo, a medida que a complexidade do desenvolvimento de software aumentou, a necessidade de se criar metodologias de desenvolvimento de software para melhorar a qualidade do produto final também cresceu. Metodologias de desenvolvimento de software, fornecem uma abordagem sistemática ao desenvolvimento de software, e é papel do engenheiro escolher a que melhor se adequá ao projeto, pois essa escolha determinará o sucesso ou não do mesmo \cite{sweebok}. Tendo visto essa ideologia e as necessidades do gestor de pessoas da empresa Zenit, nesse tópico será avaliado qual a melhor abordagem para o projeto.
\subsection{Análise de metodologia com a problemática}
A natureza de desenvolvimento de software segundo \cite{boehm2004}, se dar por meio da analise de quatro principais característica de projeto. São elas:
\begin{itemize}
    \item \textbf{Características da aplicação} – incluem objetivos primários do projeto, tamanho do projeto e o ambiente da aplicação;
    \item \textbf{Características do gerenciamento} – incluem relações com o cliente, comunicação no projeto e planejamento e controle;
    \item \textbf{Características técnicas} – incluem abordagens à definição de requisitos, desenvolvimento e teste;
    \item \textbf{Características das pessoas} – incluem características do cliente, dos desenvolvedores e da cultura da organização.
\end{itemize}
\subsubsection{Caracteristicas da aplicação}
\begin{description}
\item \textbf{Objetivos primários}

O princípio do manifesto ágil diz que mudanças nos requisitos são bem vindas a qualquer fase do desenvolvimento do software, a interação com o usuário propicia essas mudanças constantes com requisitos instáveis \cite{agileManifest}. As abordagens tradicionais defendem a previsibilidade e a estabilidade em projetos, visando bastante documentação, ou seja, elas são melhores aplicadas em projetos com requisitos estáveis, sendo menos favoráveis à mudanças  \cite{boehm2004}.

A mudança do objetivo primário do que foi conversado com o representante da empresa é razoavelmente estável, visto que este possui a necessidade de manter os dados cadastrais de seus funcionários. Entretanto a área de gestão de pessoas da empresa está implementando os processos para o gerenciamento dos funcionários, tais como: meritocracia para funcionários, bonificações, acompanhamento das atividades dos membros da empresa, e este fato interfere no levantamento dos requisitos do produto a ser desenvolvido. Logo, como há a presença de instabilidade nos processos da empresa, a metodologia ágil é a que mais se adéqua para atender essas características.

\item \textbf{Tamanho do projeto}

Atualmente, processos ágeis funcionam melhor com o número pequeno a médio de pessoas em uma equipe, ou com aplicações pequenas. O RUP exige um número maior de artefatos, formalização e por isso ele não se sobressai em relação aos métodos ágeis \cite{boehm2004}.

Este projeto possui uma pequena quantidade de envolvidos, quatro alunos da disciplina de Engenharia de Requisitos de Software levantarão os requisitos. O cliente, gestor de pessoas da Zenit, orientará e validará os requisitos do produto, a mestre da disciplina e o coach da equipe auxiliarão no projeto. Portanto, este fator direciona a escolha da metodologia ágil.
\end{description}
\subsubsection{Característica de gerenciamento}

\begin{description}
\item \textbf{Relação com o cliente}

Metodologias ágeis necessitam de uma interação bastante próxima com o cliente, fato que proporciona maior análise por parte da equipe em relação ao trabalho já feito, facilitando atingir as necessidades e requisitos esperados da aplicação de forma mais rápida, pois o cliente fornece um feedback constante do projeto, com menos necessidade de documentação, caso haja maior participação e interação dele com a equipe. Já em metodologias tradicionais, esse contato com o cliente é mais formal, visa mais documentação e menos interações cliente - servidor, possibilitando ao cliente que ele continue suas atividades rotineiras \cite{boehm2004}.

O cliente não exige muita documentação, e prefere um contato menos formal com a equipe de produção do sistema. Dessa forma, como definido por BOEHM, este contato menos formal,a maior proximidade com a equipe de desenvolvimento do software, a maior interação do cliente com ela também proporciona um feedback constante, logo a metodologia mais adequada é a ágil.

\item \textbf{Planejamento e controle}

Em metodologias ágeis, geralmente as equipes tem a possibilidade de não gastar muito tempo com planejamento, pois utilizam da interação com o cliente para gerar constantes feedbacks. Grande parte da documentação gerada pelas metodologias tradicionais, são realizadas na parte da documentação, sendo cerca de 20\% do tempo do projeto planejando e replanejando \cite{boehm2004}.

O cliente não exige muita documentação, e como os processão de gestão de pessoas da empresa estão em implementação, a metodologia tradicional é pouco recomendada, visto que gasta-se 20\% do templo planejando e replanejando e mudanças durante o processo de desenvolvimento de software ocasionariam retrabalho, este poderia ser reduzido com as metodologias ágeis.

\item \textbf{Comunicação}

O manifesto ágil defende que todos indivíduos façam parte das interações realizadas, para isso, é necessário uma comunicação bastante próxima com todos os indivíduos, fato que retira a responsabilidade de uma única equipe de planejar todo o projeto, pois esse trabalho é dividido \cite{agileManifest}. Outra grande vantagem, da forma de contato da abordagem ágil, é o desenvolvimento compartilhado e a necessidade de um conhecimento tácito, que geram uma maior adaptação da equipe às necessidades do projeto. A forma de comunicação da abordagem tradicional tende a ter sentido único dependendo fortemente de conhecimento explícito documentado, fato que dificulta uma equipe ficar por dentro do que é realizado por outra equipe \cite{boehm2004}.

O cliente tem disponibilidade de horário para reuniões e os horários são compatíveis com a equipe de engenharia de requisitos. Portanto, é facilitada a aplicação da abordagem ágil.

\end{description}

\subsubsection{Características técnicas}
\begin{description}
\item \textbf{Requisitos}

A maioria dos métodos ágeis expressam requisitos em termos ajustáveis e histórias informais, e por ter ciclos de iteração rápidos, são capazes de lidar com projeto com requisitos instáveis. Nesse tipo de abordagem, o cliente junto com a equipe de desenvolvedores se juntam com o fim de priorizar os requisitos para que possam ser executados de forma que irão gerar mais valor ao cliente. Na abordagem tradicional, o levantamento de requisitos é mais formalizado e lento, sendo mais utilizado quando há requisitos mais estáveis \cite{boehm2004}.

Como foi visto em tópicos anteriores, os requisitos podem mudar junto com a implementação dos processos da empresa. Essa mudança traz riscos aos projetos que utilizam metodologia tradicional. Além disso, o cliente tem interesse em participar com maior frequência na elicitação e desenvolvimento dos requisitos.

Desenvolvimento e teste são características que não serão levadas em consideração, pois não são o foco da disciplina.

\end{description}

\subsubsection{Características das pessoas}
\begin{description}

\item \textbf{Relacionamento com o cliente}

A maior diferença entre a metodologia ágil para a tradicional na seção “relacionamento entre o cliente”, é que na ágil, é enfatizado que o cliente faz parte da equipe, enquanto na tradicional, esse relacionamento entre cliente e desenvolvedores, é um relacionamento mais contratual \cite{boehm2004}.

O perfil do cliente, é mais proativo, e é mais fácil para ele ter controle do que será realizado no projeto, caso ele esteja mais próximo da equipe do que por via de documentos burocráticos.

\item \textbf{Desenvolvedores}

Um dos maiores problemas da comunicação na abordagem ágil, é também uma das maiores vantagens, o conhecimento tácito, mas ele só se torna problema, em equipes não maduras e/ou equipes grandes. Aplicar um desenvolvimento ágil, é relativamente mais difícil do que aplicar uma abordagem tradicional \cite{boehm2004}.

Apesar de ser mais difícil aplicar uma abordagem ágil, por necessitar uma maior maturidade \cite{boehm2004}, os integrantes da equipe estão mais acostumados a desenvolver utilizando essa metodologia de desenvolvimento, pois a equipe já desenvolveu softwares nessa metodologia e dois dos integrantes estagiam em locais onde são aplicados os conceitos ágeis para o desenvolvimento.
\end{description}
\subsection{Escolha da metodologia}

Cada característica de projeto, gerenciamento e pessoas determinam a pré disposição de uma metologia de desenvolvimento prevalecer em relação a outra para que ela seja bem executada. O framework serve como base para selecionar o mais importante do uso ferramental que ele pode oferecer, dessa forma, é importante extrair o máximo dele para que o grupo de envolvidos com o projeto consiga adaptá-lo para cada problemática. Tendo em vista essa análise, a equipe concluiu que a abordagem que mais se encaixa no contexto é a ágil do Scaled Agile Framework \cite{safe}.

\section{Processo de engenharia de requisitos}

A intenção dessa seção é esclarecer as práticas e processo de Engenharia de Requisitos. Como foi definido na seção anterior, a abordagem utilizada será a ágil.

O Framework que a equipe se baseou para modelar o processo a ser utilizado foi o Scaled Agile Framework \cite{safe}. Este framework possui uma estrutura de quatro níveis, portifólio, fluxo de valores, programa, time.

\begin{figure}[H]
    \centering
    \label{safeBigPicture}
    \includegraphics[keepaspectratio=true,scale=0.5]{figuras/safeBigPicture.eps}
    \caption[Big picture SAFe]{Big picture SAFe. fonte:\url{http://www.scaledagileframework.com}}
\end{figure}

\subsection{Nível abstrato de portifólio}

É o nível mais alto no framework, ele provê a organização básica para a organização em torno do fluxo de valores para alcançar o objetivo estratégico, agrupa pessoas e processos para este fim. O principal componente deste nível é o fluxo de valor, no qual cada um possui uma longa vida de uma série de definições, desenvolvimento, implantação de sistemas \cite{safe}.

Neste nível é feito a conexão entre a empresa cliente e a desenvolvedora. Há uma via dupla, no qual a empresa fornece os temas estratégicos, ou seja, os contextos de negócio para guiar a produção do portifólio, e na outra via a empresa recebe a resposta do contexto do portifólio com as soluções atuais contendo suas fraquezas e benefícios, novas oportunidades e possíveis ameaças \cite{safe}.

Neste nível tem-se um nível mais abstrato de requisitos, chamado de épicos de negócio (\textit{business epics}) que refletem as novas capacidades do sistema e há também os épicos habilitadores (\textit{enables epics}) que são definições técnicas arquiteturais necessárias para habilitar as novas capacidades. Estes épicos são responsabilidade dos donos dos épicos (\textit{epic owners}) e para que fique visível para todos os envolvidos no projeto, os épicos ficam em um backlog aguardando o momento para serem levados ao sistema de kanban para serem implementados \cite{safe}.

Este nível será utilizado no processo, pois é nele onde são identificados os interesses do cliente, e controlar o fluxo contínuo de valor a ser entregue a este, assim como determinar o que será realizado ou não tendo em vista os temas estratégicos da empresa.

\subsection{Nível abstrato Fluxo de valor}

Este nível é voltado para grandes construções e soluções complexas, que envolvem um grande número de colaboradores para construção multidisciplinar. É enderessado para projetos grandes e inclusive que possuam uma solução crítica. Isto exige a produção de mais artefatos, maior coordenação e que inclui um framework econômico para garantir o financiamento do fluxo de valores \cite{safe}.

Visto que o projeto da solução não é um projeto que envolve muitas áreas do conhecimento e não é um sistema crítico com soluções complexas, de larga escala e que hajam vários outros projetos interagindo junto, este nível abstrato descrito no SAFe não se adéqua ao contexto do problema da empresa júnior Zenit Aerospace. A área do conhecimento tratado pelo projeto é de administração.

\subsection{Nível abstrato Programa}

Determina os times e outros recursos que são necessários para o desenvolvimento. Há a implementação do \textit{Agile Release Train} (ART) que é uma organização falsa criada para remover passos, intervenções desnecessárias e acelerar a entrega de valor. Em geral, cada ART realiza uma parte, ou todo, o fluxo de valor e possui a infraestrutura necessária para entregar projetos de software \cite{safe}.

Neste nível há a descrição dos requisitos ainda em nível abstrato chamados de características de negócio (\textit{business features}) e características habilitadoras (\textit{enables features}) que correspondem aos seus épicos. E assim como os épicos, estes ficam visíveis em um backlog e em um kanban \cite{safe}.

Por tratar do planejamento a nível menos abstrato que o portfólio e mais que o nível do time, usaremos este nível pois ele é o intermédio dos interesses da empresa e o time que irá desenvolver os requisitos mais específicos.

\subsection{Nível abstrato Time}

Os times possuem uma organização, modelo de processo e função para desenvolvimento das atividades, em que é utilizado o \textit{ScrumXP} ou times kanban. Cada time é parte de um ART e deve definir, construir e testar as histórias de usuário (\textit{user histories}) que representam as funcionalidades do sitema como pequenas histórias e histórias habilitadoras (\textit{enable histories}) que são a descrição dos requisitos não funcionais \cite{safe}.

Este nível é o time que irá realizar a descrição mais detalhada dos requisitos e produzir o backlog do produto, logo, este não pode faltar. Como haverá apenas um ART e um time ágil desenvolvendo, determina-se que este irá adotar o \textit{Scrum}.
