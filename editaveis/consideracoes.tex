\chapter[Considera{\c c}{\~o}es finais]{Considerações finais}

A equipe considera que o objetivo final deste documento foi alcançado, sendo que esse é determinar o processo de trabalho que será adotado para a produção dos requisitos do software. Também considera que todos os tópicos exigidos neste trabalho pela disciplina de Engenharia de requisitos de software da Universidade de Brasília, foram desenvolvidos corretamente, baseados em referênciais teóricos de diferentes fontes.

Baseado no contexto da empresa júnior Zenite Aerospace, e nas características do projeto a ser desenvolvido, apresentamos uma metodologia para o desenvolvimento da solução baseado nos métodos ágeis e no framework SAFe. A definição de um processo de engenharia de requisitos baseado no framework que atenda as cinco atividades básicas de requisitos: elicitar, documentar, verificar e validar, analisar e gerênciar requisitos. Os atributos de requisitos e uma técnica para fazer a rastreabilidade dos requisitos de forma que seja possível gerência-los, para casos de mudanças dos requisitos durante o desenvolvimento do projeto. Técnicas de elicitação de requisitos, a fim de utilizar de métodos adequados de extrair os requisitos do cliente. E por final, definição de uma ferramenta de gerência de requisitos que auxilie a equipe no desenvolvimento do projeto.

Com os conteúdos apresentados neste relatório esperamos utiliza-los para construir os requisitos corretos, coerentes, completos, não-ambíguos e rastreáveis que sejam identificados a partir dos problemas da empresa júnior Zenit Aerospace e que possibilitem a construção correta de uma solução em software para a empresa.
