\chapter[Considera{\c c}{\~o}es finais]{Considerações finais}
Nessa seção será apresentado as experiências adquiridas pela equipe com a execução do trabalho, com as técnicas de elicitação utilizadas e com a disciplina de Engenharia de Requisitos.
\section{Experiência com a execução do trabalho}

Tendo como base a modelagem de processo de requisitos realizada na primeira etapa do trabalho, na segunda etapa, a equipe executou o planejamento. Dessa forma, percebemos a importância de uma boa modelagem de processo de requisitos para o sucesso de um projeto de software.

Seguindo as atividade e datas prevista no cronograma, a equipe registrou a execução das atividades elaboradas no processo de engenharia de requisitos, definido no primeiro trabalho. Assim, foi possível aprender elicitar requisitos tendo como base os princípios da metodologia ágil e do framework escolhido o SAFe, seguindo os níveis de portfólio, programa e time. Tendo como vista os três níveis já mencionados, a equipe considera que alcançou o objetivo final, sendo que foi possível levantar todas as necessidades planejadas para o escopo da disciplina.

Com a execução de uma sprint, foi possível entregar funcionalidades que agregam valor ao cliente.

\section{Experiência com as técnicas de elicitação}

As técnicas de elicitação foram essenciais na identificação dos temas estratégicos, épicos de negócio, features e história de usuários. Portanto, ganhamos experiência e maturidade para trabalhar com as técnicas escolhidas para determinadas situações que forma imaginadas na fase de desenvolvimento do processo. 

A técnica de entrevista aberta nos proporcionou entendermos melhor a problemática da empresa e enxergar com isso os temas estratégicos e épicos de negócio, nos forçando a fazer uma melhor elaboração da pauta da reunião assim como elaborar boas perguntas para que se fosse atingido o objetivo da reunião. 

O brainstorming nos proporcionou, escutar o cliente a respeito de suas ideias e sobre um determinado problema e dar abertura para novas ideias e soluções para o problema propostas pela equipe ou pelo cliente. 

A prototipagem ajudou a tornar mais claro ideias e pontos de vistas que estavam mais abstratos definindo melhor necessidades do gestor de pessoas.

\section{Experiência com a disciplina de Engenharia de Requisitos}

Após o término do trabalho da disciplina, a equipe ganhou experiência com trabalhos em equipe e aprendeu a lidar com pessoas que fazem parte do projeto, mas que não tem o conhecimento técnico que possuímos. Isso proporcionou uma maturação nesse quesito  e possibilitou a equipe se aproximar de uma equipe auto-gerenciável com uma alta qualidade de projeto.

O contato com o cliente foi tão saudável e produtivo, que a equipe se comprometeu a desenvolver parte dos requisitos levantados nas férias.
