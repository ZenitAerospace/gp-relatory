\chapter[Planejamento do projeto]{Planejamento do projeto}
\section{Cronograma}
O cronograma foi dividido de acordo com os elementos que devem ser entregues pelas atividades da engenharia de requisitos definindo também, de forma geral, a segunda etapa que será realizada na disciplina. No cronograma estão presentes as atividades que deverão ser desempenhadas pela equipe no prazo definido para entrega dos componentes deste trabalho.
Os campos presentes no cronograma são:
\begin{itemize}
    \item \textbf{Indice:} é a primeira coluna do cronograma e contém o indice das atividades, esse é ordenado de acordo com a ordem de execução das atividades.
    \item \textbf{Nome:} é um nome significativo e auto explicativo para a atividade a ser desempenhada. 
    \item \textbf{Duração:} estimativa da quantidade de dias que irá durar determinada atividade, podendo esta ser prorrogável, caso seja necessário.
    \item \textbf{Início/Fim:} campos que representam as datas previstas de execução das atividades.
    \item \textbf{Completo:} define a condição atual do desenvolvimento de uma atividade:
        \begin{itemize}
            \item 0\% não iniciada
            \item  50\% em andamento
            \item 100\% completa.
        \end{itemize}
    \item \textbf{Recursos:} estabelece o responsável ou responsáveis pela execução de determinada tarefa, não restringe os demais integrantes a participarem da execução desta.
\end{itemize}
\section{Etapa 1}
Este cronograma \ref{cronograma1} define as atividades que serão realizadas para a produção deste documento, e seus tópicos.
\begin{figure}[H]
    \centering
    \label{cronograma1}
    \includegraphics[keepaspectratio=true,scale=0.55]{figuras/cronogramaAtividades1.eps}
    \caption{Cronograma de atividades primeira etapa}
\end{figure}
\section{Etapa 2}
Este cronograma \ref{cronograma2} define um possível planejamento para a segunda etapa do trabalho, visto que como a metodologia utilizada será ágil, não é possível elaborar um planejamento detalhado para a execução das tarefas.
\begin{figure}[H]
    \centering
    \label{cronograma2}
    \includegraphics[keepaspectratio=true,scale=0.52]{figuras/cronogramaAtividades2.eps}
    \caption{Cronograma de atividades segunda etapa}
\end{figure}
