\chapter[Elicitação de requisitos]{Elicitação de requisitos}
A elicitação de requisitos contempla a área de identificar e obter informações sobre requisitos do sistema, de forma a prover o correto e mais completo entendimento do software em questão.

\section{Técnicas de elicitação}
       
Para que o processo de elicitação de requisitos seja obtenha sucesso é importante a utilização de técnicas. As técnicas a seguir foram escolhidas por se adequarem a quantidade e características de stakeholders, complexidade do contexto e objetivo da técnica. 

\subsection{Brainstorming}
Brainstorming, uma tradução literal para a língua portuguesa seria "tempestade cerebral", é uma dinâmica de grupo. É uma técnica utilizada para resolver problemas específicos, para reunir conhecimentos e gerar novas ideias ou projetos \cite{leffingwell2011}.

Essa técnica será utilizada da seguinte forma: em sua primeira etapa, utilizaremos o recurso de um facilitador, membro da equipe de requisitos, este fará valer as regras da dinâmica sendo elas: Qualquer ideia é bem vinda, ou seja, criticas serão rejeitadas; quanto mais ideias melhor, a dinâmica tem como objetivo gerar ideias para solucionar o problema em questão; Ele também será responsável também por controlar a dinâmica para que o tema em pauta não saia de foco, os temas iniciais serão levantados antes da reunião apenas com intuito de dar uma direção a técnica, pode acontecer de mais temas forem se revelando, estes serão documentados e acrescentados a pauta da dinâmica. Haverá um escrita para documentar as ideias que foram propostas e quem as propôs.

Ao término dos temas em discussão, passará para a segunda etapa da técnica, que será discutir sobres as ideias que foram levantadas, estas também serão documentadas. Cada envolvido presente na reunião terá que falar a sua opinião a respeito da ideia seja refutando,  aprimorando ou apoiando esta. O intuito dessa etapa é lapidar as ideias e excluir possíveis ideias que não se encaixem no contexto. 

\subsection{Prototipagem}
Prototipagem é uma implementação parcial de um sistema de forma rápida, podendo ser de alta ou baixa fidelidade. O intuito é elicitar requisitos que tenham um alto grau de incerteza, ou quando for necessário um rápido feedback dos usuários \cite{sommerville1995}. O protótipo pode ser usado para validar requisitos levantados com outras técnicas. Ao final da avaliação do protótipo este será descartado, ou seja, não fará parte do produto final.

Portanto, essa técnica será utilizada somente quando houver incerteza de certos requisitos, ou quando os stakeholders não conseguem expressar os requisitos, ou como forma visual de validação dos requisitos por parte do Product Onwer. Ou seja esta técnica será dependente de outra técnica de elicitação dos requisitos ela agirá como refinamento dos requisitos e validação podendo haver a descoberta de outros requisitos.    Utilizaremos um protótipo de papel de baixa fidelidade para a aplicação desta técnica.

\subsection{Entrevista}
A Entrevista é uma das técnicas tradicionais e que produz bons resultados na fase inicial de obtenção de dados \cite{leffingwell2011}. Convém que o entrevistador, membro da equipe de requisitos, dê margem ao entrevistado, o Product Owner, para expor as suas ideias. São formuladas perguntas relacionadas às partes interessadas e suas necessidades no contexto do problema a ser resolvido, e as respectivas respostas, documentadas. 

A Entrevista será tratada da seguinte forma haverá antes, a elaboração de um plano geral para a entrevista tornando a mais eficiente em relação ao tempo e desenvolvendo questões que tentam descobrir que informação o usuário do sistema está mais interessado, isso exigirá dos encarregados de tal tarefa se contextualizarem em relação a empresa e sua problemática, assim como os requisitos já elicitados e tratados até o momento. Ao término da entrevista é necessário validar se o que foi documentado pelo entrevistador corresponde com a necessidades apresentadas pelo entrevistado e se houve entendimento claro do que foi documentado. 

