\chapter[Introdução]{Introdução}

\section{Finalidade}
Apresentar um relatório da execução do processo de engenharia de requisitos elaborado no planejamento feito no trabalho 1, sobre o contexto do problema da empresa júnior Zenit Aerospace da Universidade de Brasília.

\section{Visão Geral}
Este relatório está organizado de forma a detalhar o que foi realizado em cada nível definido no processo de engenharia de requisitos. E conterá as seguintes seções:
\begin{itemize}
\item \textbf{Introdução:} 
resume a finalidade da criação deste documento, e as informações contidas em cada uma de suas partes.
\item \textbf{Execução do processo:}
detalha as mudanças feitas no planejamento anterior e um breve resumo de como foi a execução.
\item \textbf{Execução à nível de portfólio:} 
conterá o resumo das reuniões que ocorreram para a identificação dos temas de investimento e épicos, assim como os devidos artefatos
\item \textbf{Execução à nível de programa:} 
será detalhado tudo que foi realizado pela equipe de requisitos na realização do processo a nível de programa, contendo um resumo das reuniões realizadas, e todos os artefatos gerados.
\item \textbf{Execução a nível de time:} 
detalha a execução das atividades contempladas pelo processo elaborado no trabalho 01 a nível de time.
\item \textbf{Apêndice:}
será apresentado todos os artefatos gerados pela equipe a partir do processo.
\item \textbf{Considerações finais:} 
representa as considerações finais da equipe em relação ao trabalho já realizado e as expectativas para o desenvolvimento da próxima etapa.
\item \textbf{Referências:} 
contém o referencial teórico no qual a equipe se baseou para a produção do documento.
\end{itemize}
