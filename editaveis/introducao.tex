\chapter[Introdução]{Introdução}


\section{Finalidade}
Este documento tem como objetivo determinar o processo de trabalho que será adotada para a produção dos requisitos do software que solucionará o problema da empresa júnior da universidade de brasília, Zenit Aerospace.
\section{Escopo}
A Engenharia de Requisitos é o conjunto de atividades que esclareçam de forma concisa quais as necessidades o software deve suprir. Dentre este conjunto de atividades, destaca-se cinco atividades: elicitação de requisitos, análise e negociação, documentação, verificação e validação e gerência. Assim, este documento tratará do processo de engenharia de requisitos que possa atender estas cinco atividades para elicitação de requisitos para o problema de uma empresa júnior. Além disso, será proposto a utilização de uma metodologia de desenvolvimento, características dos requisitos e uso de uma ferramenta para tal.

\section{Visão Geral}
\begin{itemize}
\item \textbf{Introdução:} 
Resume a finalidade da criação deste documento, e as informações contidas em cada uma de suas partes.
\item \textbf{Contextualização da empresa:} 
Descreve de forma breve, o contexto da empresa alvo da engenharia de requisitos, Zenit, com a forma de trabalho e com o apontamento de uma visão geral do problema.
\item \textbf{Planejamento do projeto:} 
Trata do cronograma de atividades a serem desempenhadas e breve descrição das colunas do cronograma.
\item \textbf{Processo de engenharia de requisitos:} 
Engloba a justificativa da escolha de uma metodologia para desenvolvimento dos requisitos; O modelo de processo adaptado da metodologia escolhida e a descrição das atividades; Relacionamento do processo elaborado com o modelo de Melhoria de Processos de Software Brasileiro (MPS-Br).
\item \textbf{Elicitação de requisitos:} 
Essa sessão aborda as técnicas de requisitos selecionadas para atender as necessidades de levantamento de requisitos definidos no processo de engenharia de requisitos.
\item \textbf{Gerência de requisitos:} 
Contém os atributos dos requisitos e o conteúdo necessário para fazer o rastreamento e possibilitar o gerenciamento do ciclo de vida do requisito.
\item \textbf{Ferramentas de requisitos:} 
Define uma ferramenta a ser utilizada, baseada em critérios específicos escolhidos para avaliar um conjunto de três ferramentas.
\item \textbf{Considerações finais:} 
Representa as ultimas considerações da equipe em relação ao trabalho já realizado e as expectativas para o desenvolvimento da próxima etapa.
\item \textbf{Referências:} 
Contém o referencial teórico no qual a equipe se baseou para a produção dos tópicos do documento.
\end{itemize}
