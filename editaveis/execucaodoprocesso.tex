\chapter[Execução do processo]{Execução do processo}

\section{Modificação nos atributos de requisitos}
Foram feitas as seguintes mudanças:

\begin{itemize}
\item \textbf{Nível de priorização de requisitos:}
foi definido no trabalho 1 que haveria três níveis de priorização de requisitos (alta, média e baixa) sendo que a ferramenta escolhida proporciona a distinção de mais dois, totalizando cinco níveis(altíssima, alta, média, baixa, baixíssima). Por essa razão, a equipe de requisitos preferiu utilizar esse recurso oferecido pela ferramenta.
\item \textbf{Adição de identificador de requisitos:}
no trabalho 1, foi definido um identificador que serve como abreviatura para diferenciar o nível de abstração de requisitos(tema estratégico, épico, \textit{feature} e história), entretanto não foi definido como um atributo para diferenciar requisitos funcionais e não funcionais. Assim, foi definido que será adicionado ao identificado o termo \textit{business} para requisitos funcionais e \textit{enable} para não funcionais \cite{safe}.
\section{Cronograma}
Nessa seção será apresentado o cronograma das atividades que foram realizadas. 
\begin{figure}[H]
    \centering
    \label{cronogramaSegundaParte}
    \includegraphics[keepaspectratio=true,scale=0.6]{figuras/Cronograma_de_atividades_3_.eps}
    \caption[Cronograma 2]{Cronograma de atividades parte 2}
\end{figure}
\end{itemize}

