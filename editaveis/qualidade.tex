\chapter[Modelo de qualidade de processo]{Modelo de qualidade de processo}

Modelo de qualidade de software escolhido para avaliar o grau de maturidade do processo desenvolvido foi o MPS Br.

\section{O modelo}

A iniciativa de Melhoria de Processo do Software Brasileiro (MPS-Br) tem como referência os conceitos de maturidade e capacidade de processo a fim de aperfeiçoar a qualidade, produtividade e prover avaliação dos softwares \cite{mpsbr}.

O documento se baseará no modelo de refência de melhoria do processo de software (MR-MPS-SW) o qual possui os requisitos para a elaboração de um processo. Ele contém sete níveis de maturidade rotulados hierarquicamente de G a A, que são patamares de evolução dos processos \cite{mpsbr}.

Dentre os sete níveis, a presença de processos de requisitos está nos níveis G e D.
\subsection{O nível G}

Este é o nível que define a maturidade como Parcialmente Gerenciado, e define as metas para o processo de gerência de requisitos (GRE) cujo propósito é gerenciar os requisitos das partes do projeto e produto, além de identificar divergências nos relacionamentos entre os requisitos, planos de trabalho e .os produtos de trabalho \cite{mpsbr}.

Os objetivos da GRE definidos pelo MPS-Br são:

\begin{enumerate}
    \item \textbf{O entendimento dos requisitos é obtido junto aos fornecedores de requisitos:} Esta prática é definida pelas atividades “Elicitar histórias”, “Elicitar features”, “elaborar visão”, “Elicitar temas estratégicos”,”Elicitar épicos” presentes nos níveis de programa, portfólio e time.
    \item \textbf{Os requisitos são avaliados com base em critérios objetivos e um comprometimento da equipe técnica com estes requisitos é obtido:} No processo proposto, os requisitos são avaliados pelo dono do requisito e junto a equipe, feitos pelas atividades, “Analisar viabilidade de épicos”, ”Analisar features”, “Definir critério de aceitação”, “Verificar e validar histórias”.
    \item \textbf{A rastreabilidade bidirecional entre os requisitos e os produtos de trabalho é estabelecida e mantida:} Os requisitos elicitados com o cliente possuirão o atributo de origem, que define a rastreabilidade vertical, ou seja, permite identificar o relacionamento entre o produto gerado e seu requisito originário, assim como no sentido oposto desta relação.
    \item \textbf{Revisões em planos e produtos de trabalho do projeto são realizadas visando identificar e corrigir inconsistências em relação aos requisitos:} As atividades de “Verificar e validar épicos”, “Fazer retrospectiva e revisão”, “Verificar e validar histórias”, “Verificar e validar features”, “Validar temas estratégicos” permitem realizar a revisão de todo trabalho realizado, e realizar as alterações necessárias caso sejam identificadas mudanças no requisito, pelo subprocesso “gerenciar requisitos”.
    \item \textbf{Mudanças nos requisitos são gerenciadas ao longo do projeto:} As mudanças são gerenciadas em todos os níveis do processo por uma atividade paralela. Permite gerenciar quaisquer mudanças nos requisitos em qualquer um dos níveis, analisar os riscos envolvidos com a mudança e fazer as alterações.
\end{enumerate}

\subsection{O nível D}

Este nível é o que define a maturidade como Largamente Definido, e possui o propósito de estabelecer e desenvolver os requisitos (DRE) relacionados ao projeto: cliente, produto e dos componentes do produto.

Os objetivos esperados pelo DRE segundo o MPS-Br são:
\begin{enumerate}
    \item \textbf{As necessidades, expectativas e restrições do cliente, tanto do produto quanto de suas interfaces, são identificadas:} Essas necessidades são obtidas através das atividades “elicitar temas estratégicos” e “Analisar viabilidade de épicos”, fazendo o levantamento das necessidades do cliente no contexto em que ela se encaixa.
    \item \textbf{Um conjunto definido de requisitos do cliente é especificado e priorizado a partir das necessidades, expectativas e restrições identificadas:} As atividades “Priorizar épicos”, “Priorizar histórias”, “Priorizar features”, ”Executar planing poker” são responsáveis por priorizar a partir das necessidades expectativas e restrições identificadas.
    \item \textbf{Um conjunto de requisitos funcionais e não-funcionais, do produto e dos componentes do produto que descrevem a solução do problema a ser resolvido, é definido e mantido a partir dos requisitos do cliente:} As atividades de “Criar backlog de épicos”, “Montar backlog de features”, e o artefato de saída backlog da atividade “Priorizar histórias” são responsáveis por manter o conjunto de requisitos funcionais do produto e do componentes do produto.
    \item \textbf{Os requisitos funcionais e não-funcionais de cada componente do produto são refinados, elaborados e alocados:} Esses resultados são decompostos das atividades “Analisar features” e “Priorizar features”. Analisar features tem as features elaboradas e alocadas em um backlog e priorizar features refina essas features e aloca elas em roadmap.
    \item \textbf{Interfaces internas e externas do produto e de cada componente do produto são definidas:} O processo não define atividade para produzir as interfaces internas e externas do produto. Especifica apenas quais características funcionais e não funcionais este deverá ter.
    \item \textbf{Conceitos operacionais e cenários são desenvolvidos:} Os artefatos utilizados para descrever os requisitos não incluem a criação de cenários específicos para descrição dos requisitos.
    \item \textbf{Os requisitos são analisados, usando critérios definidos, para balancear as necessidades dos interessados com as restrições existentes:} O processo faz a análise de todos os requisitos extraídos do cliente, sejam eles habilitadores que representam os requisitos não-funcionais ou de negócio que representam os funcionais, em qualquer um dos três níveis, cada um teles tem atividades de verificar e validar, assim como a “Analisar features”, “Analisar viabilidade de épicos”, “Definir critérios de aceitação”.
    \item \textbf{Os requisitos são validados:} As atividades de “verificar e validar” presentes em todos os três níveis, portfólio, programa e time garantem que os requisitos sejam validados pelo cliente.
\end{enumerate}
